% Template for PLoS
% Version 3.4 January 2017
%
% % % % % % % % % % % % % % % % % % % % % %
%
% -- IMPORTANT NOTE
%
% This template contains comments intended 
% to minimize problems and delays during our production 
% process. Please follow the template instructions
% whenever possible.
%
% % % % % % % % % % % % % % % % % % % % % % % 
%
% Once your paper is accepted for publication, 
% PLEASE REMOVE ALL TRACKED CHANGES in this file 
% and leave only the final text of your manuscript. 
% PLOS recommends the use of latexdiff to track changes during review, as this will help to maintain a clean tex file.
% Visit https://www.ctan.org/pkg/latexdiff?lang=en for info or contact us at latex@plos.org.
%
%
% There are no restrictions on package use within the LaTeX files except that 
% no packages listed in the template may be deleted.
%
% Please do not include colors or graphics in the text.
%
% The manuscript LaTeX source should be contained within a single file (do not use \input, \externaldocument, or similar commands).
%
% % % % % % % % % % % % % % % % % % % % % % %
%
% -- FIGURES AND TABLES
%
% Please include tables/figure captions directly after the paragraph where they are first cited in the text.
%
% DO NOT INCLUDE GRAPHICS IN YOUR MANUSCRIPT
% - Figures should be uploaded separately from your manuscript file. 
% - Figures generated using LaTeX should be extracted and removed from the PDF before submission. 
% - Figures containing multiple panels/subfigures must be combined into one image file before submission.
% For figure citations, please use "Fig" instead of "Figure".
% See http://journals.plos.org/plosone/s/figures for PLOS figure guidelines.
%
% Tables should be cell-based and may not contain:
% - spacing/line breaks within cells to alter layout or alignment
% - do not nest tabular environments (no tabular environments within tabular environments)
% - no graphics or colored text (cell background color/shading OK)
% See http://journals.plos.org/plosone/s/tables for table guidelines.
%
% For tables that exceed the width of the text column, use the adjustwidth environment as illustrated in the example table in text below.
%
% % % % % % % % % % % % % % % % % % % % % % % %
%
% -- EQUATIONS, MATH SYMBOLS, SUBSCRIPTS, AND SUPERSCRIPTS
%
% IMPORTANT
% Below are a few tips to help format your equations and other special characters according to our specifications. For more tips to help reduce the possibility of formatting errors during conversion, please see our LaTeX guidelines at http://journals.plos.org/plosone/s/latex
%
% For inline equations, please be sure to include all portions of an equation in the math environment.  For example, x$^2$ is incorrect; this should be formatted as $x^2$ (or $\mathrm{x}^2$ if the romanized font is desired).
%
% Do not include text that is not math in the math environment. For example, CO2 should be written as CO\textsubscript{2} instead of CO$_2$.
%
% Please add line breaks to long display equations when possible in order to fit size of the column. 
%
% For inline equations, please do not include punctuation (commas, etc) within the math environment unless this is part of the equation.
%
% When adding superscript or subscripts outside of brackets/braces, please group using {}.  For example, change "[U(D,E,\gamma)]^2" to "{[U(D,E,\gamma)]}^2". 
%
% Do not use \cal for caligraphic font.  Instead, use \mathcal{}
%
% % % % % % % % % % % % % % % % % % % % % % % % 
%
% Please contact latex@plos.org with any questions.
%
% % % % % % % % % % % % % % % % % % % % % % % %

\documentclass[10pt,letterpaper]{article}
\usepackage[top=0.85in,left=2.75in,footskip=0.75in]{geometry}

% amsmath and amssymb packages, useful for mathematical formulas and symbols
\usepackage{amsmath,amssymb}

% Use adjustwidth environment to exceed column width (see example table in text)
\usepackage{changepage}

% Use Unicode characters when possible
\usepackage[utf8x]{inputenc}

% textcomp package and marvosym package for additional characters
\usepackage{textcomp,marvosym}

% cite package, to clean up citations in the main text. Do not remove.
\usepackage{cite}

% Use nameref to cite supporting information files (see Supporting Information section for more info)
\usepackage{nameref,hyperref}

% line numbers
\usepackage[right]{lineno}

% ligatures disabled
\usepackage{microtype}
\DisableLigatures[f]{encoding = *, family = * }

% color can be used to apply background shading to table cells only
\usepackage[table]{xcolor}

% array package and thick rules for tables
\usepackage{array}

% create "+" rule type for thick vertical lines
\newcolumntype{+}{!{\vrule width 2pt}}

% create \thickcline for thick horizontal lines of variable length
\newlength\savedwidth
\newcommand\thickcline[1]{%
  \noalign{\global\savedwidth\arrayrulewidth\global\arrayrulewidth 2pt}%
  \cline{#1}%
  \noalign{\vskip\arrayrulewidth}%
  \noalign{\global\arrayrulewidth\savedwidth}%
}

% \thickhline command for thick horizontal lines that span the table
\newcommand\thickhline{\noalign{\global\savedwidth\arrayrulewidth\global\arrayrulewidth 2pt}%
\hline
\noalign{\global\arrayrulewidth\savedwidth}}


% Remove comment for double spacing
%\usepackage{setspace} 
%\doublespacing

% Text layout
\raggedright
\setlength{\parindent}{0.5cm}
\textwidth 5.25in 
\textheight 8.75in

% Bold the 'Figure #' in the caption and separate it from the title/caption with a period
% Captions will be left justified
\usepackage[aboveskip=1pt,labelfont=bf,labelsep=period,justification=raggedright,singlelinecheck=off]{caption}
\renewcommand{\figurename}{Fig}

% Use the PLoS provided BiBTeX style
\bibliographystyle{plos2015}

% Remove brackets from numbering in List of References
\makeatletter
\renewcommand{\@biblabel}[1]{\quad#1.}
\makeatother

% Leave date blank
\date{}

% Header and Footer with logo
\usepackage{lastpage,fancyhdr,graphicx}
\usepackage{epstopdf}
\pagestyle{myheadings}
\pagestyle{fancy}
\fancyhf{}
\setlength{\headheight}{27.023pt}
\lhead{\includegraphics[width=2.0in]{PLOS-submission.eps}}
\rfoot{\thepage/\pageref{LastPage}}
\renewcommand{\footrule}{\hrule height 2pt \vspace{2mm}}
\fancyheadoffset[L]{2.25in}
\fancyfootoffset[L]{2.25in}
\lfoot{\sf PLOS}

%% Include all macros below

\newcommand{\lorem}{{\bf LOREM}}
\newcommand{\ipsum}{{\bf IPSUM}}

%% END MACROS SECTION


\begin{document}
\vspace*{0.2in}

% Title must be 250 characters or less.
\begin{flushleft}
{\Large
\textbf\newline{Structure-Informed Phylogenetic Analysis of the Aminoacyl-tRNA Synthetases} % Please use "sentence case" for title and headings (capitalize only the first word in a title (or heading), the first word in a subtitle (or subheading), and any proper nouns).
}
\newline
% Insert author names, affiliations and corresponding author email (do not include titles, positions, or degrees).
\\
Alex Popinga\textsuperscript{1},
Charles Carter\textsuperscript{2},
Peter Wills\textsuperscript{3*}
%Name4 Surname\textsuperscript{2},
%Name5 Surname\textsuperscript{2\ddag},
%Name6 Surname\textsuperscript{2\ddag},
%Name7 Surname\textsuperscript{1,2,3*},
%with the Lorem Ipsum Consortium\textsuperscript{\textpilcrow}
\\
\bigskip
\textbf{1} Department of Computer Science, University of Auckland, Auckland, New Zealand
\\
\textbf{2} Biochemistry and Biophysics, University of North Carolina, Chapel Hill, North Carolina, United States
\\
\textbf{3} Department of Physics, University of Auckland, Auckland, New Zealand
\\
\bigskip

% Insert additional author notes using the symbols described below. Insert symbol callouts after author names as necessary.
% 
% Remove or comment out the author notes below if they aren't used.
%
% Primary Equal Contribution Note
\Yinyang These authors contributed equally to this work.

% Additional Equal Contribution Note
% Also use this double-dagger symbol for special authorship notes, such as senior authorship.
\ddag These authors also contributed equally to this work.

% Current address notes
\textcurrency Current Address: Department of Physics, University of Auckland, Auckland, New Zealand % change symbol to "\textcurrency a" if more than one current address note
% \textcurrency b Insert second current address 
% \textcurrency c Insert third current address

% Deceased author note
%\dag Deceased

% Group/Consortium Author Note
%\textpilcrow Membership list can be found in the Acknowledgments section.

% Use the asterisk to denote corresponding authorship and provide email address in note below.
* p.wills@auckland.ac.nz

\end{flushleft}
% Please keep the abstract below 300 words
\section*{Abstract}
The aminoacyl-tRNA synthetases have had their phylogenies described previously ~\cite{bib1}.  However, for the first time we have taken into account the structural information of the proteins, using it to inform the alignments from which the phylogenies are inferred.

% Please keep the Author Summary between 150 and 200 words
% Use first person. PLOS ONE authors please skip this step. 
% Author Summary not valid for PLOS ONE submissions.   
\section*{Author summary}

\linenumbers

% Use "Eq" instead of "Equation" for equation citations.
\section*{Introduction}
The aminoacyl-tRNA synthetases, which are divided by their core structures into two distinct classes, are fundamental to understanding the origin of the genetic code.

\begin{eqnarray}
\label{eq:schemeP}
%	\mathrm{P_Y} = \underbrace{H(Y_n) - H(Y_n|\mathbf{V}^{Y}_{n})}_{S_Y} + \underbrace{H(Y_n|\mathbf{V}^{Y}_{n})- H(Y_n|\mathbf{V}^{X,Y}_{n})}_{T_{X\rightarrow Y}},
\end{eqnarray}

\section*{Materials and methods}
\subsection*{X-ray crystallography structures}



\subsection*{Modelling structures}
To ensure a statistically significant and diverse dataset, we gathered additional amino acid sequences from GenBank.  We selected 20 organisms from each of the 3 domains of life, sampling from as many distinct clades as possible, HHpred and MODELLER


\subsection*{Structural alignments}


Partial Order Structure Alignment (POSA)


\subsection*{Conserved regions}


\subsection*{Phylogenetic inference}
BEAST 2

% For figure citations, please use "Fig" instead of "Figure".
 Fig~\ref{fig1}  \nameref{S1_Video} 

% Place figure captions after the first paragraph in which they are cited.
\begin{figure}[!h]
\caption{{\bf Bold the figure title.}
Figure caption text here, please use this space for the figure panel descriptions instead of using subfigure commands. A: Lorem ipsum dolor sit amet. B: Consectetur adipiscing elit.}
\label{fig1}
\end{figure}

% Results and Discussion can be combined.
\section*{Results}
Nulla mi mi, venenatis sed ipsum varius, Table~\ref{table1} volutpat euismod diam. Proin rutrum vel massa non gravida. Quisque tempor sem et dignissim rutrum. Lorem ipsum dolor sit amet, consectetur adipiscing elit. Morbi at justo vitae nulla elementum commodo eu id massa. In vitae diam ac augue semper tincidunt eu ut eros. Fusce fringilla erat porttitor lectus cursus, vel sagittis arcu lobortis. Aliquam in enim semper, aliquam massa id, cursus neque. Praesent faucibus semper libero.

% Place tables after the first paragraph in which they are cited.
\begin{table}[!ht]
\scriptsize
\begin{adjustwidth}{-2.66in}{0in} % Comment out/remove adjustwidth environment if table fits in text column.
\centering
\caption{
{\bf Archaea aaRS data sampled and analysed}}
\begin{tabular}{|l+l|l|l|l|l|l|l|l|l|l|l|l|l|l|l|l|l|l|l|l|}
\hline
\multicolumn{1}{|l|}{\bf Organism} & \multicolumn{1}{|l|}{\bf ala} & \multicolumn{1}{|l|}{\bf arg} & \multicolumn{1}{|l|}{\bf asn} & \multicolumn{1}{|l|}{\bf asp} & \multicolumn{1}{|l|}{\bf cys} & \multicolumn{1}{|l|}{\bf gln} & \multicolumn{1}{|l|}{\bf glu} & \multicolumn{1}{|l|}{\bf gly} & \multicolumn{1}{|l|}{\bf his} & \multicolumn{1}{|l|}{\bf ile} & \multicolumn{1}{|l|}{\bf leu} & \multicolumn{1}{|l|}{\bf lys} & \multicolumn{1}{|l|}{\bf met} & \multicolumn{1}{|l|}{\bf phe} & \multicolumn{1}{|l|}{\bf pro} & \multicolumn{1}{|l|}{\bf ser} & \multicolumn{1}{|l|}{\bf thr} & \multicolumn{1}{|l|}{\bf trp} & \multicolumn{1}{|l|}{\bf tyr} & \multicolumn{1}{|l|}{\bf val} \\ \thickhline
$A.\ fulgidus$ & $\Theta$ & $\theta$ & + & $\theta$ & $\theta$ & n/a & $\theta$ & $\theta$ & $\theta$ & $\theta$ & + & $\theta$ & $\theta$ & $\theta$ & $\theta$ & $\theta$ & $\theta$ & $\theta$ & $\Theta$ & $\theta$ \\ \hline
$A.\ pernix$ & $\theta$ & $\theta$ & -- & $\theta$ & $\theta$ & n/a & $\theta$ & $\theta$ & $\theta$ & $\theta$ & $\theta$ & $\theta$ & $\theta$ & $\theta$ & $\theta$ & $\theta$ & $\Theta$ & -- & $\Theta$ & $\theta$ \\ \hline
$Halobacterium\ sp.$ & + & $\theta$ & + & $\theta$ & $\theta$ & n/a & $\theta$ & $\theta$ & $\theta$ & $\theta$ & + & $\theta$ & $\theta$ & $\theta$ & + & $\theta$ & $\theta$ & $\theta$ & $\theta$ & $\theta$ \\ \hline
$M.\ acetivorans$ & $\theta$ & $\theta$ & + & $\theta$ & + & n/a & + & $\theta$ & $\theta$ & $\theta$ & + & $\theta$ & $\theta$ & $\theta$ & $\theta$ & $\theta$ & $\theta$ & $\theta$ & $\theta$ & $\theta$ \\ \hline
$M.\ aeolicus\ Nankai$ & $\theta$ & $\theta$ & + & $\theta$ & + & n/a & + & $\theta$ & $\theta$ & + & + & $\theta$ & $\theta$ & $\theta$ & $\theta$ & $\theta$ & $\theta$ & + & $\theta$ & $\theta$ \\ \hline
$M.\ barkeri$ & -- & -- & -- & -- & -- & n/a & -- & -- & -- & -- & -- & -- & -- & -- & -- & $\Theta$ & -- & -- & -- & -- \\ \hline
$M.\ hungatei$ & $\theta$ & $\theta$ & + & + & $\theta$ & n/a & $\theta$ & $\theta$ & $\theta$ & $\theta$ & + & + & $\theta$ & $\theta$ & $\theta$ & $\theta$ & $\theta$ & $\theta$ & $\theta$ & $\theta$ \\ \hline
$M.\ jannaschii$ & $\theta$ & $\theta$ & + & $\theta$ & -- & n/a & $\theta$ & $\theta$ & $\theta$ & $\theta$ & + & -- & $\theta$ & $\theta$ & $\Theta$ & $\theta$ & $\theta$ & $\theta$ & $\Theta$ & $\theta$ \\ \hline
$M.\ kandleri$ & $\theta$ & $\theta$ & + & $\theta$ & + & n/a & $\theta$ & $\theta$ & $\theta$ & $\theta$ & + & $\theta$ & $\theta$ & $\theta$ & $\theta$ & $\theta$ & $\theta$ & $\theta$ & $\theta$ & $\theta$ \\ \hline
$M.\ thermautotrophicus$ & $\theta$ & $\theta$ & + & $\theta$ & + & n/a & $\theta$ & $\theta$ & $\theta$ & $\theta$ & $\theta$ & $\theta$ & $\theta$ & $\theta$ & + & $\theta$ & $\theta$ & $\theta$ & $\theta$ & $\theta$ \\ \hline
$N.\ equitans$ & + & $\theta$ & $\theta$ & $\theta$ & $\theta$ & n/a & + & $\theta$ & $\theta$ & $\theta$ & $\theta$ & $\theta$ & $\theta$ & $\theta$ & $\theta$ & $\theta$ & $\theta$ & $\theta$ & $\theta$ & $\theta$ \\ \hline
$P.\ abyssi$ & -- & -- & -- & -- & -- & n/a & -- & -- & -- & -- & -- & -- & $\Theta$ & -- & -- & -- & -- & -- & -- & -- \\ \hline
$P.\ aerophilum$ & + & $\theta$ & $\theta$ & $\theta$ & $\theta$ & n/a & $\theta$ & $\theta$ & $\theta$ & $\theta$ & $\theta$ & + & $\theta$ & $\theta$ & $\theta$ & $\theta$ & $\theta$ & $\theta$ & $\theta$ & $\theta$ \\ \hline
$P.\ delaneyi$ & $\theta$ & $\theta$ & -- & $\theta$ & $\theta$ & n/a & $\theta$ & $\theta$ & $\theta$ & $\theta$ & $\theta$ & $\theta$ & $\theta$ & $\theta$ & $\theta$ & $\theta$ & $\theta$ & $\theta$ & $\theta$ & $\theta$ \\ \hline
$P.\ horikoshii$ & $\Theta$ & $\Theta$ & $\Theta$ & $\theta$ & $\theta$ & n/a & + & $\theta$ & $\theta$ & $\theta$ & $\Theta$ & $\Theta$ & $\theta$ & $\theta$ & $\theta$ & $\theta$ & $\theta$ & $\theta$ & $\Theta$ & $\theta$ \\ \hline
$P.\ occultum$ & $\theta$ & $\theta$ & -- & -- & $\theta$ & n/a & $\theta$ & $\theta$ & $\theta$ & $\theta$ & $\theta$ & $\theta$ & $\theta$ & $\theta$ & $\theta$ & $\theta$ & $\theta$ & $\theta$ & $\theta$ & $\theta$ \\ \hline
$S.\ acidocaldarius$ & $\theta$ & $\theta$ & + & $\theta$ & $\theta$ & n/a & $\theta$ & $\theta$ & $\theta$ & $\theta$ & $\theta$ & + & $\theta$ & + & $\theta$ & $\theta$ & $\theta$ & $\theta$ & $\theta$ & $\theta$ \\ \hline
$S.\ marinus$ & $\theta$ & $\theta$ & $\theta$ & $\theta$ & $\theta$ & n/a & $\theta$ & $\theta$ & -- & $\theta$ & $\theta$ & $\theta$ & $\theta$ & $\theta$ & $\theta$ & $\theta$ & $\theta$ & + & $\theta$ & + \\ \hline
$S.\ tokodaii$ & -- & -- & -- & $\Theta$ & -- & n/a & -- & -- & -- & -- & -- & -- & -- & -- & -- & -- & -- & -- & -- & -- \\ \hline
$T.\ acidophilum$ & + & $\theta$ & $\theta$ & $\theta$ & $\theta$ & n/a & $\theta$ & $\theta$ & $\Theta$ & $\theta$ & $\theta$ & $\theta$ & $\theta$ & + & $\theta$ & $\theta$ & $\theta$ & $\theta$ & $\theta$ & $\theta$ \\ \hline
$T.\ kodakarensis$ & $\theta$ & $\theta$ & $\theta$ & $\Theta$ & $\theta$ & n/a & $\theta$ & $\theta$ & $\theta$ & $\theta$ & $\theta$ & $\theta$ & $\theta$ & $\theta$ & $\theta$ & $\theta$ & $\theta$ & $\theta$ & $\theta$ & + \\ \hline
$T.\ volcanium$ & $\theta$ & + & $\theta$ & $\theta$ & + & n/a & $\theta$ & $\theta$ & $\theta$ & $\theta$ & $\theta$ & $\theta$ & $\theta$ & $\theta$ & $\theta$ & $\theta$ & $\theta$ & $\theta$ & $\theta$ & $\theta$ \\ \hline
\end{tabular}
\begin{flushleft} \textbf{$\Theta$} means we have the crystal structures for this organism; $\theta$ means the data was collected from Genbank, successfully modelled, and aligned; + means the data was collected from Genbank but not used in the final alignment; -- means the data was not obtained
\end{flushleft}
\label{table1}
\end{adjustwidth}
\end{table}

\begin{table}[!ht]
\scriptsize
\begin{adjustwidth}{-2.66in}{0in} % Comment out/remove adjustwidth environment if table fits in text column.
\centering
\caption{
{\bf Bacteria aaRS data sampled and analysed}}
\begin{tabular}{|l+l|l|l|l|l|l|l|l|l|l|l|l|l|l|l|l|l|l|l|l|}
\hline
\multicolumn{1}{|l|}{\bf Organism} & \multicolumn{1}{|l|}{\bf ala} & \multicolumn{1}{|l|}{\bf arg} & \multicolumn{1}{|l|}{\bf asn} & \multicolumn{1}{|l|}{\bf asp} & \multicolumn{1}{|l|}{\bf cys} & \multicolumn{1}{|l|}{\bf gln} & \multicolumn{1}{|l|}{\bf glu} & \multicolumn{1}{|l|}{\bf gly} & \multicolumn{1}{|l|}{\bf his} & \multicolumn{1}{|l|}{\bf ile} & \multicolumn{1}{|l|}{\bf leu} & \multicolumn{1}{|l|}{\bf lys} & \multicolumn{1}{|l|}{\bf met} & \multicolumn{1}{|l|}{\bf phe} & \multicolumn{1}{|l|}{\bf pro} & \multicolumn{1}{|l|}{\bf ser} & \multicolumn{1}{|l|}{\bf thr} & \multicolumn{1}{|l|}{\bf trp} & \multicolumn{1}{|l|}{\bf tyr} & \multicolumn{1}{|l|}{\bf val} \\ \thickhline
$A.\ aeolicus$ & $\Theta$ & $\theta$ & + & $\theta$ & $\theta$ & + & + & + & $\theta$ & $\theta$ & $\theta$ & + & $\Theta$ & $\theta$ & $\theta$ & $\theta$ & $\theta$ & $\theta$ & $\theta$ & + \\ \hline
$B.\ burgdorferi$ & $\theta$ & $\theta$ & $\theta$ & $\theta$ & $\theta$ & + & $\theta$ & $\theta$ & $\theta$ & $\theta$ & $\theta$ & + & $\theta$ & + & $\theta$ & $\theta$ & $\theta$ & $\theta$ & $\theta$ & $\theta$ \\ \hline
$B.\ fragilis$ & $\theta$ & $\theta$ & $\theta$ & $\theta$ & $\theta$ & $\theta$ & $\theta$ & $\theta$ & $\theta$ & $\theta$ & + & $\theta$ & $\theta$ & + & $\theta$ & $\theta$ & $\theta$ & $\theta$ & $\theta$ & + \\ \hline
$B.\ licheniformis$ & $\theta$ & $\theta$ & + & $\theta$ & $\theta$ & + & $\theta$ & + & $\theta$ & $\theta$ & $\theta$ & $\theta$ & $\theta$ & $\theta$ & $\theta$ & $\theta$ & $\theta$ & $\theta$ & $\theta$ & $\theta$ \\ \hline
$B.\ thailandensis$ & $\theta$ & $\theta$ & + & $\theta$ & $\theta$ & $\theta$ & $\theta$ & $\theta$ & $\theta$ & $\theta$ & $\theta$ & $\theta$ & $\theta$ & $\theta$ & $\theta$ & $\theta$ & + & $\theta$ & + & $\theta$ \\ \hline
$C.\ A.\ asiaticus$ & $\theta$ & $\theta$ & $\theta$ & $\theta$ & $\theta$ & + & + & $\theta$ & $\theta$ & $\theta$ & $\theta$ & $\theta$ & $\theta$ & $\theta$ & $\theta$ & $\theta$ & $\theta$ & $\theta$ & $\theta$ & $\theta$ \\ \hline
$C.\ aggregans$ & $\theta$ & $\theta$ & $\theta$ & $\theta$ & $\theta$ & + & $\theta$ & $\theta$ & $\theta$ & $\theta$ & $\theta$ & $\theta$ & $\theta$ & $\theta$ & $\theta$ & $\theta$ & $\theta$ & $\theta$ & $\theta$ & + \\ \hline
$C.\ jejuni$ & + & $\Theta$ & + & $\theta$ & $\theta$ & $\theta$ & $\theta$ & + & $\theta$ & $\theta$ & $\theta$ & $\theta$ & $\theta$ & $\theta$ & + & $\theta$ & $\theta$ & $\theta$ & $\theta$ & $\theta$ \\ \hline
$C.\ thermalis$ & $\theta$ & $\theta$ & $\theta$ & + & $\theta$ & + & + & + & $\theta$ & $\theta$ & $\theta$ & + & $\theta$ & $\theta$ & $\theta$ & $\theta$ & $\theta$ & $\theta$ & $\theta$ & $\theta$ \\ \hline
$D.\ radiodurans$ & $\theta$ & $\theta$ & $\theta$ & $\theta$ & $\theta$ & $\Theta$ & $\theta$ & $\theta$ & $\theta$ & $\theta$ & $\theta$ & $\theta$ & $\theta$ & $\theta$ & $\theta$ & $\theta$ & $\theta$ & $\theta$ & $\theta$ & $\theta$ \\ \hline
$E.\ coli$ & $\Theta$ & $\theta$ & $\theta$ & $\Theta$ & $\Theta$ & $\Theta$ & $\Theta$ & $\Theta$ & $\Theta$ & $\theta$ & $\theta$ & $\Theta$ & $\Theta$ & $\theta$ & $\theta$ & $\theta$ & $\Theta$ & $\theta$ & $\Theta$ & $\theta$ \\ \hline
$E.\ faecalis$ & $\theta$ & $\theta$ & $\theta$ & $\theta$ & $\theta$ & -- & $\theta$ & $\theta$ & $\theta$ & $\theta$ & $\theta$ & $\theta$ & $\theta$ & $\theta$ & $\Theta$ & $\theta$ & $\theta$ & $\theta$ & $\theta$ & $\theta$ \\ \hline
$G.\ obscuriglobus$ & $\theta$ & $\theta$ & $\theta$ & + & $\theta$ & $\theta$ & $\theta$ & $\theta$ & $\theta$ & $\theta$ & $\theta$ & $\theta$ & $\theta$ & $\theta$ & $\theta$ & + & $\theta$ & $\theta$ & $\theta$ & + \\ \hline
$G.\ stearothermophilus$ & $\theta$ & $\theta$ & + & $\theta$ & + & $\theta$ & $\theta$ & $\theta$ & $\theta$ & $\theta$ & $\theta$ & $\Theta$ & $\theta$ & -- & $\theta$ & $\theta$ & $\theta$ & $\Theta$ & $\theta$ & $\theta$ \\ \hline
$H.\ aurantiacus$ & $\theta$ & $\theta$ & $\theta$ & $\theta$ & $\theta$ & -- & $\theta$ & $\theta$ & $\theta$ & $\theta$ & + & $\theta$ & $\theta$ & $\theta$ & $\theta$ & $\theta$ & $\theta$ & $\theta$ & $\theta$ & + \\ \hline
$M.\ mobile$ & $\theta$ & + & $\theta$ & $\theta$ & $\theta$ & -- & $\Theta$ & $\theta$ & $\theta$ & $\theta$ & $\theta$ & $\theta$ & + & $\theta$ & + & $\theta$ & + & + & $\theta$ & $\theta$ \\ \hline
$M.\ smegmatis$ & + & + & + & $\theta$ & $\Theta$ & + & + & + & + & + & $\theta$ & + & $\theta$ & $\theta$ & + & $\theta$ & + & + & + & $\theta$ \\ \hline
$P.\ mikurensis$ & $\theta$ & $\theta$ & -- & $\theta$ & $\theta$ & $\theta$ & $\theta$ & $\theta$ & $\theta$ & $\theta$ & + & $\theta$ & $\theta$ & $\theta$ & $\theta$ & + & $\theta$ & $\theta$ & $\theta$ & + \\ \hline
$R.\ marinus$ & + & $\theta$ & $\theta$ & $\theta$ & $\theta$ & -- & $\theta$ & $\theta$ & $\theta$ & $\theta$ & + & + & $\theta$ & $\theta$ & $\theta$ & $\theta$ & $\theta$ & $\theta$ & * & $\theta$ \\ \hline
$R.\ palustris$ & $\theta$ & $\theta$ & -- & $\theta$ & $\theta$ & $\theta$ & + & $\theta$ & $\theta$ & + & + & $\theta$ & $\theta$ & $\theta$ & $\Theta$ & $\theta$ & $\theta$ & $\theta$ & $\theta$ & $\theta$ \\ \hline
$S.\ aureus$ & $\theta$ & $\theta$ & $\theta$ & $\theta$ & $\theta$ & + & + & $\Theta$ & $\Theta$ & $\Theta$ & $\theta$ & $\theta$ & $\theta$ & $\theta$ & $\theta$ & $\theta$ & $\Theta$ & $\theta$ & $\Theta$ & $\theta$ \\ \hline
$S.\ elongatus$ & $\theta$ & $\theta$ & $\theta$ & $\theta$ & $\theta$ & $\theta$ & $\Theta$ & $\theta$ & $\theta$ & $\theta$ & $\theta$ & $\theta$ & $\theta$ & + & $\theta$ & $\theta$ & $\theta$ & $\theta$ & $\theta$ & $\theta$ \\ \hline
$S.\ haemolyticus$ & $\theta$ & $\theta$ & $\theta$ & $\theta$ & $\theta$ & -- & $\theta$ & $\theta$ & $\theta$ & $\theta$ & $\theta$ & $\theta$ & $\theta$ & $\Theta$ & $\theta$ & $\theta$ & $\theta$ & $\theta$ & $\theta$ & $\theta$ \\ \hline
$S.\ typhimurium$ & -- & -- & -- & -- & -- & -- & -- & -- & -- & -- & -- & $\Theta$ & -- & -- & -- & -- & -- & -- & -- & -- \\ \hline
$T.\ maritima$ & $\theta$ & $\theta$ & + & $\theta$ & $\theta$ & $\Theta$ & $\Theta$ & $\Theta$ & $\theta$ & $\theta$ & $\theta$ & $\theta$ & $\theta$ & -- & $\theta$ & $\theta$ & $\theta$ & $\Theta$ & $\theta$ & $\theta$ \\ \hline
$T.\ thermophilus$ & $\theta$ & $\Theta$ & $\Theta$ & $\Theta$ & $\theta$ & $\Theta$ & $\Theta$ & $\Theta$ & $\Theta$ & $\Theta$ & $\Theta$ & $\theta$ & $\Theta$ & $\Theta$ & $\Theta$ & $\Theta$ & $\theta$ & $\Theta$ & $\Theta$ & $\Theta$ \\ \hline
\end{tabular}
\begin{flushleft} \textbf{$\Theta$} means we have the crystal structures for this organism; $\theta$ means the data was collected from Genbank, successfully modelled, and aligned; + means the data was collected from Genbank but not used in the final alignment; -- means the data was not obtained
\end{flushleft}
\label{table1}
\end{adjustwidth}
\end{table}

\begin{table}[!ht]
\scriptsize
\begin{adjustwidth}{-2.56in}{0in} % Comment out/remove adjustwidth environment if table fits in text column.
\centering
\caption{
{\bf Eukaryote aaRS data sampled and analysed}}
\begin{tabular}{|l+l|l|l|l|l|l|l|l|l|l|l|l|l|l|l|l|l|l|l|l|}
\hline
\multicolumn{1}{|l|}{\bf Organism} & \multicolumn{1}{|l|}{\bf ala} & \multicolumn{1}{|l|}{\bf arg} & \multicolumn{1}{|l|}{\bf asn} & \multicolumn{1}{|l|}{\bf asp} & \multicolumn{1}{|l|}{\bf cys} & \multicolumn{1}{|l|}{\bf gln} & \multicolumn{1}{|l|}{\bf glu} & \multicolumn{1}{|l|}{\bf gly} & \multicolumn{1}{|l|}{\bf his} & \multicolumn{1}{|l|}{\bf ile} & \multicolumn{1}{|l|}{\bf leu} & \multicolumn{1}{|l|}{\bf lys} & \multicolumn{1}{|l|}{\bf met} & \multicolumn{1}{|l|}{\bf phe} & \multicolumn{1}{|l|}{\bf pro} & \multicolumn{1}{|l|}{\bf ser} & \multicolumn{1}{|l|}{\bf thr} & \multicolumn{1}{|l|}{\bf trp} & \multicolumn{1}{|l|}{\bf tyr} & \multicolumn{1}{|l|}{\bf val} \\ \thickhline
$A.\ subglabra$ & -- & $\theta$ & $\theta$ & $\theta$ & -- & $\theta$ & $\theta$ & $\theta$ & $\theta$ & $\theta$ & $\theta$ & $\theta$ & $\theta$ & + & $\theta$ & $\theta$ & $\theta$ & $\theta$ & $\theta$ & -- \\ \hline
$B.\ hominis$ & $\theta$ & -- & -- & $\theta$ & $\theta$ & -- & $\theta$ & $\theta$ & -- & $\theta$ & $\theta$ & $\theta$ & $\theta$ & $\theta$ & $\theta$ & -- & -- & -- & $\theta$ & $\theta$ \\ \hline
$B.\ mori$ & $\theta$ & $\Theta$ & $\theta$ & $\theta$ & -- & -- & $\theta$ & $\theta$ & -- & $\theta$ & -- & $\theta$ & $\theta$ & $\theta$ & + & $\theta$ & $\theta$ & $\theta$ & -- & -- \\ \hline
$B.\ taurus$ & + & + & + & -- & + & -- & + & + & + & + & $\theta$ & + & + & + & + & $\Theta$ & + & $\theta$ & + & + \\ \hline
$C.\ gigas$ & $\theta$ & + & $\theta$ & $\theta$ & $\theta$ & $\theta$ & + & $\theta$ & $\theta$ & $\theta$ & $\theta$ & $\theta$ & $\theta$ & $\theta$ & $\theta$ & $\theta$ & $\theta$ & $\theta$ & $\theta$ & $\theta$ \\ \hline
$C.\ owczarzaki$ & $\theta$ & $\theta$ & $\theta$ & $\theta$ & $\theta$ & $\theta$ & + & $\theta$ & $\theta$ & $\theta$ & $\theta$ & $\theta$ & + & + & $\theta$ & $\theta$ & $\theta$ & $\theta$ & $\theta$ & $\theta$ \\ \hline
$C.\ parvum\ Iowa\ II$ & $\theta$ & -- & $\theta$ & $\theta$ & $\theta$ & $\theta$ & $\theta$ & $\theta$ & $\theta$ & $\theta$ & -- & $\theta$ & $\theta$ & + & $\theta$ & $\theta$ & + & $\theta$ & $\theta$ & $\theta$ \\ \hline
$C.\ subellipsoidea$ & $\theta$ & $\theta$ & + & $\theta$ & -- & $\theta$ & -- & $\theta$ & + & $\theta$ & $\theta$ & $\theta$ & $\theta$ & + & $\theta$ & $\theta$ & $\theta$ & + & $\theta$ & $\theta$ \\ \hline
$D.\ discoideum$ & $\theta$ & $\theta$ & $\theta$ & $\theta$ & $\theta$ & -- & + & $\theta$ & $\theta$ & $\theta$ & + & $\theta$ & $\theta$ & $\theta$ & $\theta$ & $\theta$ & $\theta$ & $\theta$ & $\theta$ & $\theta$ \\ \hline
$D.\ rerio$ & $\theta$ & $\theta$ & + & -- & + & $\theta$ & $\theta$ & + & $\theta$ & + & + & -- & + & + & + & + & + & $\theta$ & + & $\theta$ \\ \hline
$E.\ guttata$ & $\theta$ & + & + & $\theta$ & $\theta$ & $\theta$ & $\theta$ & -- & $\theta$ & $\theta$ & $\theta$ & + & $\theta$ & $\theta$ & $\theta$ & $\theta$ & $\theta$ & $\theta$ & + & $\theta$ \\ \hline
$E.\ histolytica$ & $\theta$ & + & $\Theta$ & $\Theta$ & $\theta$ & $\theta$ & $\theta$ & $\theta$ & $\theta$ & $\theta$ & + & -- & $\theta$ & $\theta$ & $\theta$ & $\theta$ & $\theta$ & $\theta$ & $\theta$ & $\theta$ \\ \hline
$G.\ lamblia$ & + & $\theta$ & $\theta$ & $\theta$ & $\theta$ & $\theta$ & -- & $\theta$ & $\theta$ & $\theta$ & $\theta$ & $\theta$ & $\theta$ & $\theta$ & $\Theta$ & $\theta$ & $\theta$ & $\theta$ & $\theta$ & $\theta$ \\ \hline
$G.\ theta$ & $\theta$ & $\theta$ & $\theta$ & $\theta$ & $\theta$ & $\theta$ & + & $\theta$ & $\theta$ & $\theta$ & $\theta$ & $\theta$ & $\theta$ & $\theta$ & $\theta$ & $\theta$ & $\theta$ & $\theta$ & $\theta$ & + \\ \hline
$H.\ sapiens$ & + & + & $\theta$ & -- & -- & $\Theta$ & $\theta$ & $\Theta$ & $\theta$ & $\theta$ & $\theta$ & $\Theta$ & $\theta$ & $\Theta$ & + & $\theta$ & $\theta$ & $\Theta$ & $\theta$ & -- \\ \hline
$L.\ infantum$ & $\theta$ & $\theta$ & $\theta$ & $\theta$ & $\theta$ & $\theta$ & $\theta$ & $\theta$ & $\theta$ & $\theta$ & $\theta$ & $\theta$ & $\theta$ & $\theta$ & $\theta$ & $\theta$ & $\theta$ & $\theta$ & $\theta$ & $\theta$ \\ \hline
$M.\ domestica$ & + & $\theta$ & + & + & + & $\theta$ & + & + & $\theta$ & + & + & + & + & $\theta$ & + & + & + & $\theta$ & $\theta$ & $\theta$ \\ \hline
$N.\ ceranae$ & $\theta$ & $\theta$ & $\theta$ & $\theta$ & $\theta$ & $\theta$ & $\theta$ & $\theta$ & + & $\theta$ & $\theta$ & $\theta$ & $\theta$ & + & $\theta$ & $\theta$ & $\theta$ & $\theta$ & $\theta$ & $\theta$ \\ \hline
$N.\ gruberi$ & $\theta$ & -- & $\theta$ & $\theta$ & $\theta$ & -- & -- & $\theta$ & -- & $\theta$ & + & $\theta$ & $\theta$ & $\theta$ & $\theta$ & $\theta$ & $\theta$ & $\theta$ & $\theta$ & $\theta$ \\ \hline
$P.\ bivittatus$ & + & + & + & + & $\theta$ & + & + & + & + & + & + & + & + & + & + & + & -- & + & + & + \\ \hline
$P.\ chromatophora$ & $\theta$ & $\theta$ & -- & $\theta$ & $\theta$ & -- & $\theta$ & + & $\theta$ & $\theta$ & $\theta$ & $\theta$ & $\theta$ & $\theta$ & $\theta$ & $\theta$ & $\theta$ & $\theta$ & $\theta$ & $\theta$ \\ \hline
$P.\ euphratica$ & + & $\theta$ & $\theta$ & -- & + & + & -- & -- & + & + & + & + & + & + & + & + & + & + & + & + \\ \hline
$S.\ cerevisiae$ & -- & $\Theta$ & -- & -- & -- & -- & -- & -- & -- & -- & -- & -- & -- & -- & -- & -- & -- & $\Theta$ & $\Theta$ & -- \\ \hline
$S.\ rosetta$ & $\theta$ & $\theta$ & $\theta$ & $\theta$ & $\theta$ & $\theta$ & $\theta$ & $\theta$ & -- & $\theta$ & + & $\theta$ & $\theta$ & $\theta$ & $\theta$ & $\theta$ & $\theta$ & + & $\theta$ & $\theta$ \\ \hline
$R.\ filosa$ & + & -- & + & -- & + & + & + & + & -- & -- & + & + & + & $\theta$ & $\theta$ & + & $\theta$ & + & -- & -- \\ \hline
$T.\ brucei$ & -- & -- & -- & -- & -- & -- & -- & -- & $\Theta$ & -- & -- & + & + & -- & -- & -- & -- & + & + & -- \\ \hline
$T.\ cruzi$ & -- & -- & -- & -- & -- & -- & -- & -- & $\Theta$ & -- & -- & -- & -- & -- & -- & -- & -- & -- & -- & -- \\ \hline
$T.\ vaginalis$ & + & -- & $\theta$ & -- & $\theta$ & $\theta$ & $\theta$ & $\theta$ & $\theta$ & $\theta$ & $\theta$ & $\theta$ & -- & $\theta$ & $\theta$ & $\theta$ & $\theta$ & $\theta$ & -- & + \\ \hline
\end{tabular}
\begin{flushleft} \textbf{$\Theta$} means we have the crystal structures for this organism; $\theta$ means the data was collected from Genbank, successfully modelled, and aligned; + means the data was collected from Genbank but not used in the final alignment; -- means the data was not obtained
\end{flushleft}
\label{table1}
\end{adjustwidth}
\end{table}



NOTE TO SELF: CHECK N. EQUITANS DATA

CHECK T. BRUCEI TYR/MET SEQUENCE??

%PLOS does not support heading levels beyond the 3rd (no 4th level headings).
\subsection*{\lorem\ and \ipsum\ nunc blandit a tortor}
\subsubsection*{3rd level heading} 
Maecenas convallis mauris sit amet sem ultrices gravida. Etiam eget sapien nibh. Sed ac ipsum eget enim egestas ullamcorper nec euismod ligula. Curabitur fringilla pulvinar lectus consectetur pellentesque. Quisque augue sem, tincidunt sit amet feugiat eget, ullamcorper sed velit. Sed non aliquet felis. Lorem ipsum dolor sit amet, consectetur adipiscing elit. Mauris commodo justo ac dui pretium imperdiet. Sed suscipit iaculis mi at feugiat. 

\begin{enumerate}
	\item{react}
	\item{diffuse free particles}
	\item{increment time by dt and go to 1}
\end{enumerate}


\subsection*{Sed ac quam id nisi malesuada congue}

Nulla mi mi, venenatis sed ipsum varius, volutpat euismod diam. Proin rutrum vel massa non gravida. Quisque tempor sem et dignissim rutrum. Lorem ipsum dolor sit amet, consectetur adipiscing elit. Morbi at justo vitae nulla elementum commodo eu id massa. In vitae diam ac augue semper tincidunt eu ut eros. Fusce fringilla erat porttitor lectus cursus, vel sagittis arcu lobortis. Aliquam in enim semper, aliquam massa id, cursus neque. Praesent faucibus semper libero.

\begin{itemize}
	\item First bulleted item.
	\item Second bulleted item.
	\item Third bulleted item.
\end{itemize}

\section*{Discussion}
Nulla mi mi, venenatis sed ipsum varius, Table~\ref{table1} volutpat euismod diam. Proin rutrum vel massa non gravida. Quisque tempor sem et dignissim rutrum. Lorem ipsum dolor sit amet, consectetur adipiscing elit. Morbi at justo vitae nulla elementum commodo eu id massa. In vitae diam ac augue semper tincidunt eu ut eros. Fusce fringilla erat porttitor lectus cursus, vel sagittis arcu lobortis. Aliquam in enim semper, aliquam massa id, cursus neque. Praesent faucibus semper libero~\cite{bib3}.

\section*{Conclusion}

%CO\textsubscript{2} Maecenas convallis mauris sit amet sem ultrices gravida. Etiam eget sapien nibh. Sed ac ipsum eget enim egestas ullamcorper nec euismod ligula. Curabitur fringilla pulvinar lectus consectetur pellentesque. Quisque augue sem, tincidunt sit amet feugiat eget, ullamcorper sed velit. 

Sed non aliquet felis. Lorem ipsum dolor sit amet, consectetur adipiscing elit. Mauris commodo justo ac dui pretium imperdiet. Sed suscipit iaculis mi at feugiat. Ut neque ipsum, luctus id lacus ut, laoreet scelerisque urna. Phasellus venenatis, tortor nec vestibulum mattis, massa tortor interdum felis, nec pellentesque metus tortor nec nisl. Ut ornare mauris tellus, vel dapibus arcu suscipit sed. Nam condimentum sem eget mollis euismod. Nullam dui urna, gravida venenatis dui et, tincidunt sodales ex. Nunc est dui, sodales sed mauris nec, auctor sagittis leo. Aliquam tincidunt, ex in facilisis elementum, libero lectus luctus est, non vulputate nisl augue at dolor. For more information, see \nameref{S1_Appendix}.

\section*{Supporting information}

% Include only the SI item label in the paragraph heading. Use the \nameref{label} command to cite SI items in the text.
\paragraph*{S1 Fig.}
\label{S1_Fig}
{\bf Bold the title sentence.} Add descriptive text after the title of the item (optional).

\paragraph*{S2 Fig.}
\label{S2_Fig}
{\bf Lorem ipsum.} Maecenas convallis mauris sit amet sem ultrices gravida. Etiam eget sapien nibh. Sed ac ipsum eget enim egestas ullamcorper nec euismod ligula. Curabitur fringilla pulvinar lectus consectetur pellentesque.

\paragraph*{S1 File.}
\label{S1_File}
{\bf Lorem ipsum.}  Maecenas convallis mauris sit amet sem ultrices gravida. Etiam eget sapien nibh. Sed ac ipsum eget enim egestas ullamcorper nec euismod ligula. Curabitur fringilla pulvinar lectus consectetur pellentesque.

\paragraph*{S1 Video.}
\label{S1_Video}
{\bf Lorem ipsum.}  Maecenas convallis mauris sit amet sem ultrices gravida. Etiam eget sapien nibh. Sed ac ipsum eget enim egestas ullamcorper nec euismod ligula. Curabitur fringilla pulvinar lectus consectetur pellentesque.

\paragraph*{S1 Appendix.}
\label{S1_Appendix}
{\bf Lorem ipsum.} Maecenas convallis mauris sit amet sem ultrices gravida. Etiam eget sapien nibh. Sed ac ipsum eget enim egestas ullamcorper nec euismod ligula. Curabitur fringilla pulvinar lectus consectetur pellentesque.

\paragraph*{S1 Table.}
\label{S1_Table}
{\bf Lorem ipsum.} Maecenas convallis mauris sit amet sem ultrices gravida. Etiam eget sapien nibh. Sed ac ipsum eget enim egestas ullamcorper nec euismod ligula. Curabitur fringilla pulvinar lectus consectetur pellentesque.

\section*{Acknowledgments}
Cras egestas velit mauris, eu mollis turpis pellentesque sit amet. Interdum et malesuada fames ac ante ipsum primis in faucibus. Nam id pretium nisi. Sed ac quam id nisi malesuada congue. Sed interdum aliquet augue, at pellentesque quam rhoncus vitae.

\nolinenumbers

% Either type in your references using
% \begin{thebibliography}{}
% \bibitem{}
% Text
% \end{thebibliography}
%
% or
%
% Compile your BiBTeX database using our plos2015.bst
% style file and paste the contents of your .bbl file
% here. See http://journals.plos.org/plosone/s/latex for 
% step-by-step instructions.
% 
\begin{thebibliography}{10}

\bibitem{bib1}
Conant GC, Wolfe KH.
\newblock {{T}urning a hobby into a job: how duplicated genes find new
  functions}.
\newblock Nat Rev Genet. 2008 Dec;9(12):938--950.

\bibitem{bib2}
B. Webb, A. Sali.
\newblock Comparative Protein Structure Modeling Using Modeller.
\newblock Current Protocols in Bioinformatics, John Wiley and Sons, Inc., 5.6.1-5.6.32, 2014.

\bibitem{bib3}
M.A. Marti-Renom, A. Stuart, A. Fiser, R. Sánchez, F. Melo, A. Sali. 
\newblock Comparative protein structure modeling of genes and genomes.
\newblock Annu. Rev. Biophys. Biomol. Struct. 29, 291-325, 2000.

\bibitem{bib4}
A. Sali, T.L. Blundell.
\newblock Comparative protein modelling by satisfaction of spatial restraints.
\newblock J. Mol. Biol. 234, 779-815, 1993.

\bibitem{bib5}
A. Fiser, R.K. Do, A. Sali.
\newblock Modeling of loops in protein structures.
\newblock Protein Science 9. 1753-1773, 2000.

\bibitem{bib6}
Söding J, Biegert A, Lupas AN.
\newblock The HHpred interactive server for protein homology detection and structure prediction. 
\newblock Nucleic Acids Research. 2005;33(Web Server issue):W244-W248. doi:10.1093/nar/gki408.

\bibitem{bib7}
Z Li, P Natarajan, Y Ye, T Hrabe, A Godzik.
\newblock POSA: a user-driven, interactive multiple protein structure alignment server.
\newblock Nucl. Acids Res. 2014. doi: 10.1093/nar/gku394

\end{thebibliography}


%$Aquifex aeolicus$ & ala & arg & asn & asp & cys & glu & gln & gly & his & ile & leu & lys & met & phe & pro & ser & thr & trp & tyr & val \\ \hline
%$Bacillus licheniformis$ & ala & arg & asn & asp & cys & glu & gln & gly & his & ile & leu & lys & met & phe & pro & ser & thr & trp & tyr & val \\ \hline
%$Borrelia burgdorferi$ & ala & arg & asn & asp & cys & glu & gln & gly & his & ile & leu & lys & met & phe & pro & ser & thr & trp & tyr & val \\ \hline
%$Burkholderia thailandensis$ & ala & arg & asn & asp & cys & glu & gln & gly & his & ile & leu & lys & met & phe & pro & ser & thr & trp & tyr & val \\ \hline
%$Campylobacter jejuni$ & ala & arg & asn & asp & cys & glu & gln & gly & his & ile & leu & lys & met & phe & pro & ser & thr & trp & tyr & val \\ \hline
%$Deinococcus radiodurans$ & ala & arg & asn & asp & cys & glu & gln & gly & his & ile & leu & lys & met & phe & pro & ser & thr & trp & tyr & val \\ \hline
%$Escherichia coli$ & ala & arg & asn & asp & cys & glu & gln & gly & his & ile & leu & lys & met & phe & pro & ser & thr & trp & tyr & val \\ \hline
%$Geobacillus stearothermophilus$ & ala & arg & asn & asp & cys & glu & gln & gly & his & ile & leu & lys & met & phe & pro & ser & thr & trp & tyr & val \\ \hline
%$Mycobacterium smegmatis$ & ala & arg & asn & asp & cys & glu & gln & gly & his & ile & leu & lys & met & phe & pro & ser & thr & trp & tyr & val \\ \hline
%$Mycoplasm mobile$ & ala & arg & asn & asp & cys & glu & gln & gly & his & ile & leu & lys & met & phe & pro & ser & thr & trp & tyr & val \\ \hline
%$Staphylococcus aureus$ & ala & arg & asn & asp & cys & glu & gln & gly & his & ile & leu & lys & met & phe & pro & ser & thr & trp & tyr & val \\ \hline
%$Synechococcus elongatus$ & ala & arg & asn & asp & cys & glu & gln & gly & his & ile & leu & lys & met & phe & pro & ser & thr & trp & tyr & val \\ \hline
%$Thermotoga maritima$ & ala & arg & asn & asp & cys & glu & gln & gly & his & ile & leu & lys & met & phe & pro & ser & thr & trp & tyr & val \\ \hline
%$Thermus thermophilus$ & ala & arg & asn & asp & cys & glu & gln & gly & his & ile & leu & lys & met & phe & pro & ser & thr & trp & tyr & val \\ \hline
%$Chroococcidiopsis thermalis$ & ala & arg & asn & asp & cys & glu & gln & gly & his & ile & leu & lys & met & phe & pro & ser & thr & trp & tyr & val \\ \hline
%$Candidatus Amoebophilus asiaticus$ & ala & arg & asn & asp & cys & glu & gln & gly & his & ile & leu & lys & met & phe & pro & ser & thr & trp & tyr & val \\ \hline
%$Chloroflexus aggregans$ & ala & arg & asn & asp & cys & glu & gln & gly & his & ile & leu & lys & met & phe & pro & ser & thr & trp & tyr & val \\ \hline
%$Herpetosiphon aurantiacus$ & ala & arg & asn & asp & cys & glu & gln & gly & his & ile & leu & lys & met & phe & pro & ser & thr & trp & tyr & val \\ \hline
%$Phycisphaera mikurensis$ & ala & arg & asn & asp & cys & glu & gln & gly & his & ile & leu & lys & met & phe & pro & ser & thr & trp & tyr & val \\ \hline
%$Gemmata obscuriglobus$ & ala & arg & asn & asp & cys & glu & gln & gly & his & ile & leu & lys & met & phe & pro & ser & thr & trp & tyr & val \\ \hline
%$Bacteroides fragilis$ & ala & arg & asn & asp & cys & glu & gln & gly & his & ile & leu & lys & met & phe & pro & ser & thr & trp & tyr & val \\ \hline




\end{document}

